\documentclass{article}
\title{Physics 247 HW 12}
\author{Name: Hleb Zhynko}
\date{Due: December 1, 2022}

\usepackage{amsmath}

\begin{document}

\maketitle

\section*{Problem 1}
Let $E_{initial}$ be the initial energy of the $\pi^0$, and $E_p$ - the final energy of each of the photons. Since the $\pi^0$ has no momentum in its rest frame, its total energy will be equal to its rest energy, that is, $E_{initial} = 134.9766\,MeV$. After the decay, the total energy will be conserved. Therefore, the final energy of the photons will be:

\[
E_{initial} = 2E_p;
\]
\[
E_p = 0.5 \times E_{initial} = 67.4883\,MeV.
\]

In the lab frame, the momentum of the $\pi^0$ will be conserved after the decay, so each of the photons will have $p_x = 0.5 \times 60\,\frac{GeV}{c} = 30,000\,\frac{MeV}{c}$. Since photons have no mass, $p_y = \frac{E_p}{c} = 67.4883\,\frac{MeV}{c}$. Therefore, the angle $\theta$ between the two photons will be:

\[
\theta =2\arctan{\frac{p_y}{p_x}} = 2\arctan{\frac{67.4883}{30,000}} = 2 \times 0.1288929^{\circ} = 0.2577859^{\circ}.
\]

Answering the final question, since the detector can detect photons $>= 0.1^{\circ}$ apart, it should be able to detect the two photons.

\end{document}