\documentclass{article}
\title{Physics 247 HW 10}
\author{Name: Hleb Zhynko}
\date{Due: November 18, 2022}

\usepackage{amsmath}

\begin{document}

\maketitle

\section*{Problem 1}
Let $u = 0.6c$ be the particle's velocity relative to the lab and $E_{rest} = 0.511\,MeV$ - the particle's rest energy.

\subsection*{Part a}

\[
\gamma = \frac{1}{\sqrt{1 - \frac{u}{c}^2}} = \frac{1}{\sqrt{1 - 0.6^2}} = \frac{1}{0.8} = 1.25.
\]

\subsection*{Part b}

\[
p = \gamma mu = \gamma u \frac{E_{rest}}{c^2} = 1.25 \times 0.6 \times 0.511 \frac{MeV}{c} = 0.383\,\frac{MeV}{c}.
\]

\subsection*{Part c}
Using the invariant quantity:

\[
E^2 = (pc)^2 + (mc^2)^2 = 0.383^2 + 0.511^2 = 0.40781\,MeV^2;
\]
\[
E = 0.6386\,MeV.
\]

\subsection*{Part d}

\[
E_k = E - E_{rest} = 0.6386 - 0.511 = 0.1276\,MeV.
\]

\section*{Problem 2}
Let $m_0 = 1\,kg$ be the rest mass of the iron slugs, $(u_1, u_2, u_3) = (0.9c, 0.99c, 0.999c)$ - the initial velocities of the slugs, and $M = 45 \times 10^6\,kg$ - the mass of the battleship.

\subsection*{Part a}
Let $E_{rest} = m_0c^2$ be the rest mass of the slugs, $E$ - the relativistic energy of the slugs, and $E_k$ - the kinetic energy of the slugs. Let $\beta$ denote $\frac{u}{c}$. Using the invariant quantity:

\[
E^2 = (pc)^2 + E_{rest}^2 = (\gamma m_0uc)^2 + E_{rest}^2 = E_{rest}^2 (\frac{\gamma ^2 u^2}{c^2} + 1) = E_{rest}^2(\frac{\beta ^2}{1 - \beta ^2} + 1);
\]
\[
E = E_{rest} \sqrt{\frac{\beta ^2}{1 - \beta ^2} + 1};
\]
\[
E_k = E - E_{rest};
\]
\[
E_{rest} = 1 \times (3 \times 10^8)^2 = 9 \times 10^{16}\,J\:or\:5.61798\times10^{26}\,GeV.
\]

For $u_1 = 0.9c$ ($\beta^2 = 0.81$):
\[
E = 5.61798 \sqrt{\frac{0.81}{1-0.81} + 1} \times 10^{26} = 12.88853\times10^{26}\,GeV;
\]
\[
or\:12.88853\times10^{26} \times 1.6022 \times 10^{-10} = 20.65\times10^{16}\,J.
\]
\[
E_k = 12.88853\times10^{26} - 5.61798\times10^{26} = 7.27055\times10^{26}\,GeV;
\]
\[
or\:7.27055\times10^{26} \times 1.6022 \times 10^{-10} = 11.64888\times10^{16}\,J.
\]

For $u_2 = 0.99c$ ($\beta^2 = 0.9801$):
\[
E = 5.61798 \sqrt{\frac{0.9801}{1-0.9801} + 1} \times 10^{26} = 39.8248\times10^{26}\,GeV;
\]
\[
or\:39.8248\times10^{26} \times 1.6022 \times 10^{-10} = 63.807\times10^{16}\,J.
\]
\[
E_k = 39.8248\times10^{26} - 5.61798\times10^{26} = 34.20682\times10^{26}\,GeV;
\]
\[
or\:34.20682\times10^{26} \times 1.6022 \times 10^{-10} = 54.806167\times10^{16}\,J.
\]

For $u_3 = 0.999c$ ($\beta^2 = 0.998$):
\[
E = 5.61798 \sqrt{\frac{0.998}{1-0.998} + 1} \times 10^{26} = 125.62185\times10^{26}\,GeV;
\]
\[
or\:125.62185\times10^{26} \times 1.6022 \times 10^{-10} = 201.27\times10^{16}\,J.
\]
\[
E_k = 125.62185\times10^{26} - 5.61798\times10^{26} = 120.0\times10^{26}\,GeV;
\]
\[
or\:120.0\times10^{26} \times 1.6022 \times 10^{-10} = 192.27\times10^{16}\,J.
\]

\subsection*{Part b}
Let $v$ be the final velocity of the battleship. Using the conservation of momentum:

\[
p_{initial} = p_{final};
\]
\[
0 = \gamma m_0u + Mv;
\]
\[
v = - \frac{\gamma m_0u}{M}.
\]

For $u_1 = 0.9c$ ($\gamma = 2.2942$):
\[
v \approx -13.765\,m/s.
\]

For $u_2 = 0.99c$ ($\gamma = 7.0889$):
\[
v \approx -46.7867\,m/s.
\]

For $u_3 = 0.999c$ ($\gamma = 22.3663$):
\[
v \approx -148.9596\,m/s.
\]

\subsection*{Part c}
Let $m_S = 50\,kg$ be the mass of the Sith warrior and $v$ - her resulting velocity. According to the conservation of momentum:

\[
v = \frac{\gamma m_0u}{m_S}.
\]

For $u = 0.9c$ ($\gamma = 2.2942$):
\[
v \approx 1.2388\times10^7 \,m/s.
\]

Therefore, if the warrior were to start trying to stop the slugs, she would start rapidly accelerating to $\approx 0.04c$, which, as one could imagine, would not be comfortable for a human. Consequently, we can make a conclusion that this would not be a good idea.

\section*{Problem 3}
Let $u$ be the desired speed of the particle. The following amount of kinetic energy will be required to accelerate the particle to $u$:

\[
E_k = mc^2(\gamma - 1).
\]

\subsection*{Part a}
For $u = 0.5c$, $\gamma \approx 1.1547$:
\[
E_k \approx 0.1547\,mc^2.
\]

\subsection*{Part b}
For $u = 0.9c$, $\gamma \approx 2.29416$:
\[
E_k \approx 1.29416\,mc^2.
\]

\subsection*{Part c}
For $u = 0.99c$, $\gamma \approx 7.08881$:
\[
E_k \approx 6.08881\,mc^2.
\]

\section*{Problem 4}
Let $u = 2.5 \times 10^5\,m/s$ be the Sun's velocity, and $m_0$ - its rest mass.

\subsection*{Part a}
\[
\frac{p_{relativistic}}{p_{Newtonian}} = \frac{\gamma m_0u}{m_0 u} = \gamma = \frac{1}{\sqrt{1 - (\frac{2.5\times10^5}{3\times10^8})^2}} = 1.00000035.
\]

\subsection*{Part b}
\[
\frac{KE_{relativistic}}{KE_{Newtonian}} = \frac{m_0c^2(\gamma - 1)}{\frac{1}{2}m_0u^2} = \frac{2c^2}{u^2}(\gamma - 1) = 1.008.
\]

\section*{Problem 5}
Let $m_e$, $m_\mu$ be the masses of the particles, and $u$ and $v$ - the velocities of the electron and the muon respectively in the lab frame. Let $c=1$. Since the kinetic energy of the electron is $5m_ec^2$, we can find $\gamma_u$ (the Lorentz factor for the electron's velocity):
\[
5m_e = m_e\gamma_u - m_e \Rightarrow \gamma_u = 6.
\]

We can find $u$ from $\gamma_u$:
\[
\gamma_u = \frac{1}{\sqrt{1 - u^2}};
\]
\[
u = \sqrt{1 - \frac{1}{\gamma_u^2}} \approx 0.986013.
\]

Next, according to the problem statement:
\[
KE_e = KE_\mu;
\]
\[
m_e(\gamma_u - 1) = m_\mu(\gamma_v - 1);
\]
\[
\gamma_v = \frac{m_e}{m_\mu}(\gamma_u - 1) + 1.
\]

Similarly to $u$, we can find $v$ from $\gamma_v$:
\[
\gamma_v = \frac{1}{\sqrt{1 - v^2}};
\]
\[
u = \sqrt{1 - \frac{1}{\gamma_v^2}} \approx 0.216018422.
\]

Let us transform $u$ to the muon's frame of reference using the relativistic velocity addition formula:
\[
u\prime = \frac{u-v}{1-uv} = \frac{-0.986013 - 0.216018422}{1 - (0.216018422)(-0.986013)} \approx -0.99095995.
\]

The x-component of the four-momentum vector of the electron in the muon's frame of reference would then be:
\[
p_1\prime = \gamma_{u\prime} u\prime m_e \approx -3.77444\,\frac{MeV}{c^2}.
\]

The transformed $p_0$ can be found using momentum transformation:
\[
p_0\prime = \gamma_v(p_0 - vp_1) = \gamma_v(\gamma_u m_e - v \gamma_u u m_e) = \gamma_v \gamma_u m_e (1 - uv);
\]
\[
p_0\prime \approx 3.80894\,\frac{MeV}{c^2}.
\]

Therefore, the four-momentum vector of the electron in the muon's rest frame can be expressed as the following:
\[
P_e\prime = (3.80894, -3.77444, p_2, p_3)\,\frac{MeV}{c^2}.
\]

\end{document}